\documentclass[UTF8]{ctexart}
\usepackage{amsmath}
\usepackage{amsfonts}

\title{关于数列的一些学习记录}
\author{NeterOster}
\date{\today}

\def\inlinedisplay#1{$\displaystyle #1$}

\begin{document}
	\maketitle
	\section{数列的概念}
	\paragraph{数列}
	数列是由\textbf{数字组成的序列},一个数列可以用 $\{a_n\}$ 表示,数列中的每个数字,称为数列的\textbf{项},用 $a_n$ 表示 $\{a_n\}$ 数列的第 $n$ 项.
	\paragraph{数列的通项公式}
	若能找到一个表达式 \inlinedisplay{f(n) = a_n (n \in \mathbb{Z} ^ +)} 给出数列 $\{a_n\}$ 的第 $n$ 项的值,则称这个 $f(n)$ 为数列 $\{a_n\}$ 的通项公式.
	\paragraph{数列的前$n$项和}
	一般地,用 $S_n$ 表示一个数列的前 $n$ 项和. 有 
	\[S_n=a_1+a_2+a_3+\cdots+a_n=\sum_{i=1}^n a_i\]
	特殊地,$S_1=a_1$
	\paragraph{数列的递推公式}
	数列的递推公式表示数列相邻项的关系. 例如 \inlinedisplay{a_{n+1}-a_n = d}
	
	\section{基本数列}
	\subsection{等差数列}
	\paragraph{等差数列的概念}
	等差数列,也就是递推公式可以为 \inlinedisplay{a_{n+1}-a_n=d (d \ne 0)} 的数列,它的后一项与前一项的差值是一个相等常数.
	\paragraph{等差数列的通项公式}
	等差数列的通项公式是
	\[a_n = a_1 + d(n-1) (n \in \mathbb{Z} ^ +)\]
	其中,$a_1$ 称为\textbf{首项},$d$ 称为\textbf{公差}.
	\paragraph{等差数列通项公式的推导}
	下面,我们试着推导等差数列的通项公式.
	\newline
	首先,根据等差数列的递推公式 \inlinedisplay{a_{n+1}-a_n = d},把 $n$ 取遍 \inlinedisplay{1,2,3,\cdots,n} 代入,得到一系列等式
	\begin{gather*}
	a_2-a_1=d
	\\
	a_3-a_2=d
	\\
	a_4-a_3=d
	\\
	\vdots
	\\
	a_n-a_{n-1}=d
	\end{gather*}
	将这一系列等式全部相加,我们得到 $a_n-a_1=d(n-1)$,也就是
	\[a_n = a_1 + d(n-1)\]
	这就是等差数列的通项公式.
	\paragraph{等差数列通项公式的推广}
	等差数列通项公式有一个更广泛的表达,也就是
	\[a_n = a_m + d(n-m) (n,m \in \mathbb{Z} ^ +)\]
	对于这个公式,本文不做证明.
	\paragraph{等差数列的前$n$项和}
	等差数列前$n$项和的公式是
	\[S_n = \sum_{i=1}^{n} [a_1 + d(n-1)] = \frac{a_1+a_n}{2} \cdot n\]
	关于这个公式的推导,有很多方式,本文叙述两种常用方法
	\subparagraph{倒序相加法}
	这是相当正统的一个手法. 首先
	\[S_n = a_1 + (a_1 + d) + (a_1 + 2d) + \cdots + a_1 + d(n-1)\]
	而同时有
	\[S_n = [a_1 + d(n-1)] + [a_1 + d(n-2)] + \cdots + a_1\]
	两式相加,就得到了
	\[2S_n = n(a_1 + a_n)\]
	因此
	\[S_n = \frac{a_1+a_n}{2} \cdot n \]
	\subparagraph{积分求和法}
	欲求 \inlinedisplay{f(n) = \sum_{i=1}^{n} [a_1 + d(n-1)]}
	首先求导,得到
	\[f'(n) = \sum_{i=1}^{n} d + C = nd + C\]
	再对 $f'(n)$ 积分,得到
	\[\int{f'(n)} = \frac{1}{2}dn^2 + Cn = f(n) \]
	而 \inlinedisplay{f(1) = a = \frac{1}{2}d + C},代入之,得到
	\[f(n) = \frac{1}{2}dn^2 - \frac{1}{2}dn + a \]
	这就完成了求和
	
	\paragraph{等差数列的性质}
	等差数列有许多基本性质,接下来介绍其中的几个
	\subparagraph{平均数与数值替换}
	在等差数列 $ \{a_n\} $ 中,任意的四个正整数 $m,n,p,q$ 满足 $m+n=p+q$,那么有
	\[a_m+a_n=a_p+a_q\]
	等式两边项数只要相等,且下标相加相等,那么这个等式就成立,例如
	\[a_m+a_n+a_h+a_g = a_q + a_t + a_s + a_l\]
	当 $m+n+a+g = q+t+s+l$,且它们均为正整数时成立
	\subparagraph{组成新数列}
	在等差数列 $\{a_n\} $ 中,取连续相等数目的项相加,形成新的数列 $\{b_n\}$,那么 $\{b_n\}$ 也是等差数列.
	\par
	例如 $a_1+a_2=b_1$,$a_3+a_4=b_2$,$a_5+a_6=b_3$ 等等组成的数列 $\{b_n\}$ 也是等差数列
	\par
	这种情况下,新的数列 $\{b_n\}$ 的公差 $d'$ 是 $d' = m^2 d$
	\par
	类似地,在等差数列 $\{a_n\}$ 中,隔连续相等数目的项取项作为新数列 $\{b_n\}$ 的项,那么 $\{b_n\}$ 也是等差数列
	\par
	例如,$a_1=b_1$,$a_3=b_2$,$a_5=b_3$,等等,则 $\{b_n\}$ 是等差数列
	\subparagraph{奇数项(和)与偶数项(和)}
	这里简单地抛出两个公式
	\[S_e - S_o = \frac{1}{2}nd (n \in 2k, k \in \mathbb{Z} ^ +)\]
	\[\frac{S_o}{(n+1)/2} = \frac{S_e}{(n-1)/2} (n \in 2k-1, k \in \mathbb{Z} ^ +)\]
\end{document}
